% Edgar Fournival <contact@edgar-fournival.fr>

\documentclass[11pt,a4paper]{article}
\usepackage[left=2cm,right=2cm,top=2cm,bottom=2cm]{geometry}

\usepackage[english]{babel}

\usepackage{ifxetex}

\ifxetex
  \usepackage{fontspec}
\else
  \usepackage[T1]{fontenc}
  \usepackage[utf8]{inputenc}
\fi

\setlength\parindent{0pt}
\setlength\parskip{0.25em}

\let\emptyset\varnothing
\let\leq\leqslant
\let\geq\geqslant

\usepackage{xspace}
\newcommand{\rpi}{Raspberry Pi\xspace}

\usepackage{etoolbox}
\makeatletter
\preto{\@verbatim}{\topsep=0.25em \partopsep=0.25em }
\makeatother

\newcommand{\plugmodule}[2]{Plug the \texttt{#1} pin of the module into the \texttt{#2} pin of the GPIO extension board.}
\newcommand{\plugcolumn}[4]{Plug a #1 cable between the \texttt{#2} pin and the \texttt{#3} column on the #4.}

\begin{document}

Edgar Fournival \hfill M2 COMASIC

\begin{center}
  \medskip
  \LARGE\textbf{Project report: Internet of Things}
  \medskip
\end{center}

\section*{Introduction}

\section{Setup}

\subsection{Networking}

During the development and the testing of the project, I used an Ethernet cable with a static networking configuration.

The \rpi, my home computer and the Internet connection were all connected to a Dlink switch.

In order to define a non-volatile network configuration on the \rpi, the file \texttt{/etc/network/\allowbreak interfaces} was edited by adding the following instructions:
\begin{verbatim}
iface eth0 inet static
  address 192.168.1.11
  netmask 255.255.255.0
  gateway 192.168.1.1
\end{verbatim}

My Internet connection is backed by an Orange Livebox which is on a 192.168.1.0/24 network. It supports DHCP but I preferred to give the \rpi a static configuration in order to ease the configuration of the SSH link between the devices.

\subsection{SSH}

SSH is a secure transport protocol designed to replace Telnet. It can be used to send commands to a remote device or setup things such as port forwarding, encrypted tunnels, etc. I have decided to setup SSH on the \rpi in order to drop the need to use the QWERTY keyboard, the mouse and another screen.

The SSH server service has been enabled on the \rpi, meaning it will start during the boot. This has been achieved with this command:
\begin{verbatim}
sudo systemctl enable ssh
\end{verbatim}

After that, I created a new SSH key on my home computer without a passphrase using \texttt{ssh-keygen} with its default options. The key was then sent to the \rpi using:
\begin{verbatim}
ssh-copy-id -i ~/.ssh/rpi pi@192.168.1.11
\end{verbatim}

To make things easier, I also defined a SSH alias in \verb|~/.ssh/config|:
\begin{verbatim}
Host rpi
  IdentityFile ~/.ssh/rpi
  User pi
  HostName 192.168.1.11
\end{verbatim}

This enabled me to connect to the \rpi under the alias \texttt{rpi} and without providing a password. From now I can type directly \texttt{ssh rpi} or use \texttt{rpi} in the \texttt{rsync} or \texttt{scp} commands.

\subsection{Hardware}

\subsubsection{The basics}

\begin{enumerate}
  \item	Open the \rpi in order to access the GPIO pins.
  \item	Connect the GPIO ``rainbow pride'' cable to the \rpi.
  \item	Connect the GPIO extension board to the other end of the GPIO cable.
  \item	Plug the GPIO extension board into the middle of the breadboard.
  \item	\plugcolumn{red}{5V0}{+}{right}
  \item	\plugcolumn{black}{GND}{-}{right}
  \item	\plugcolumn{red}{3V3}{+}{left}
  \item	\plugcolumn{black}{GND}{-}{left}
\end{enumerate}

\subsubsection{RGB LED module}

\begin{enumerate}
  \item	Plug the \texttt{VCC} pin of the module into the \texttt{5V0 +} column on the right.
  \item	\plugmodule{R}{GPIO18}
  \item	\plugmodule{G}{GPIO24}
  \item	\plugmodule{B}{GPIO21}
  \item	The GPIO pin has been chosen to be on the same PWM channel.
\end{enumerate}

\subsection{Software}

First, we need to update the package cache in order to be able to install new packages:
\begin{verbatim}
sudo apt-get update
\end{verbatim}

\subsubsection{The Go toolchain}

Then, we will need the Go toolchain for local testing. Note that this is useful for development purposes only (i.e. running \texttt{go test} on modules). Here is the command:
\begin{verbatim}
sudo apt-get install golang
\end{verbatim}

After that, we need to setup Go environment in a specific folder and permanently export the \texttt{GOPATH} environment variable:
\begin{verbatim}
mkdir -p ~/go && echo "export GOPATH=~/go" >> ~/.bashrc && source ~/.bashrc
\end{verbatim}

\section*{Conclusion}

%il faut expliquer l'architecture avec des flèches, en détails
%par exemple :
%ordinateur
%rpi, capteurs
%sockets, direction des flux
%activateurs (lampe, alarme, ...)
%serveurs, stockage

\end{document}
